The following is a CV written in latex using the moderncv package. Convert it to a webpage using html and css. 
Use the fontawesome library for icons and put everything in a max-width .container with a box-shadow.

About the cvitem / cventry lines: 
1. Use css grid layout to create two columns, put the first argument of cvitem / cventry into the first and the rest in the second.
2. The last argument becomes a p.description in a smaller font.
3. The other arguments are wrapped in semantic inline element.  

```
\documentclass{moderncv}        

\moderncvstyle{classic}
\moderncvcolor{red}

\name{Mathijs}{Henquet}
\phone[mobile]{+31 (0)6 39 56 53 54}
\email{mathijs.henquet@gmail.com}
\social[github]{mathijshenquet}
\social[linkedin]{mathijshenquet}

\begin{document}
\makecvtitle
\vspace*{-3em}

\section{Technical skills}
\cvitem{Advanced}{Rust, C\# {\small (.NET Core)}, Javascript {\small (Typescript, Node.js)}, React, Web {\small (HTML, CSS)}, Haskell, Mathematica, \LaTeX}
\cvitem{Intermediate}{Git, Java, PHP {\small (Symfony, Doctrine)}, SQL {\small (MySQL, PostgreSQL, Oracle)}, NoSQL {\small (MongoDB, Redis)}, Docker, Linux, Powershell, Bash}
\cvitem{Basic}{\textit{\small Many more...}}

\section{Education}
\cventry{2020 -- \phantom{2020}}{Master Mathematics}{Utrecht University}{}{Ongoing}{}
\cventry{2015 -- 2019}{Bachelor Informatics}{Utrecht University}{}{Grade point average 7.8\textcolor{gray}{/10}}{}
\cventry{2012 -- 2019}{Bachelor of Mathematics}{Utrecht University}{}{Grade point average 7.8\textcolor{gray}{/10}}{}
\cventry{2016 -- 2017}{Erasmus semester}{Freiburg University, Germany}{}{}{}
\cventry{2006 -- 2012}{Secondary education}{De Nieuwste School, Tilburg}{}{Grade point average 8.5\textcolor{gray}{/10}}{}

\section{Projects and awards}
\cventry{2022--\phantom{0}}{\normalfont Master thesis project}{}{}{Ongoing}{On the correctness of automatic differentiation in the presence of recursion and iteration. }
\cventry{2019}{\normalfont Summer fellow at MIRI}{}{}{}{An extended retreat by the Machine Intelligence Research Institution for mathematicians and programmers with a serious interest in making technical progress on the problem of AI alignment.}
\cventry{2019}{\normalfont \href{http://mathijshenquet.nl/cv/henquet_homotopical_mathematics.pdf}{Bachelor Thesis Homotopical mathematics \textcolor{color2}{\faIcon[regular]{file}}}}{}{}{Grade: 9\textcolor{gray}{/10}}{On homotopy type theory and its relation to higher topos theory.}
\cventry{2018}{\normalfont \href{http://mathijshenquet.nl/cv/kieli_software_project.pdf}{Software project Speech2EPD \textcolor{color2}{\faIcon[regular]{file}}}}{}{}{}{We made a system which produces a electronic medical file by collecting multimodal inputs such as recorded consults and digital measurements. For this I designed and implemented a flexible microservices based architecture which served as the backbone of this system. }
\cventry{2016}{\normalfont Winner of $\mu$KP programming competition}{}{}{}{A programming competition amongst the CS students at Utrecht University.}

\section{Experience}
\cventry{2020}{\normalfont \href{https://github.com/epidemics/covid}{Volunteer at Epidemic Forecasting \textcolor{color2}{\faIcon[regular]{github}}}}{}{}{}{Supporting decision makers with Covid19 scenarios and modeling, geared toward the global south. I worked on presenting the models in the frontend. }
\cventry{2015}{\normalfont Tutor Linear Algebra at Utrecht University}{}{}{}{}

\section{Language skills}
\begin{minipage}[t]{0.33\textwidth}
\cvitem{Dutch}{C2 - Native}
\end{minipage}
\begin{minipage}[t]{0.33\textwidth}
\cvitem{English}{C2 - Near native}
\end{minipage}
\begin{minipage}[t]{0.33\textwidth}
\cvitem{German}{C1 - Excellent}
\end{minipage}

\end{document}
```